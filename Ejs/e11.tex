\item Sea $(V,\langle\cdot,\cdot\rangle)$ un espacio vectorial con producto interno y sea $W$ un subespacio de $V$.
    \begin{enumerate}
        \item Si $v=w+w^\prime$ con $w\in W$ y $w^\prime\in W^\perp$, probar que $J:V\to V$ dado por \[J(v)=w-w^\prime,\]
            es un operador lineal autoadjunto y unitario.
            \begin{mdframed}[style=s]
                
            \end{mdframed}
        \item Consideremos $V=\R^3$ dotado con el producto interno usual y $W=\overline{\{(1,0,1)\}}$.
            \begin{enumerate}
                \item Hallar la representación matricial de $J$ en la base canónica de $\R^3$.
                    \begin{mdframed}[style=s]
                        
                    \end{mdframed}
                \item Hallar una base ortonormal de autovectores de $J$.
                    \begin{mdframed}[style=s]
                        
                    \end{mdframed}
            \end{enumerate}
    \end{enumerate}